\documentclass[conference]{IEEEtran}
\IEEEoverridecommandlockouts

% ==========================================
% PACKAGES
% ==========================================
\usepackage{cite}
\usepackage{amsmath,amssymb,amsfonts}
\usepackage{algorithmic}
\usepackage{algorithm}
\usepackage{graphicx}
\usepackage{textcomp}
\usepackage{xcolor}
\usepackage{booktabs}
\usepackage{multirow}
\usepackage{listings}
\usepackage{url}
\usepackage{float}

% --- TIKZ & PLOTS ---
\usepackage{tikz}
\usepackage{pgfplots}
\pgfplotsset{compat=1.17}
\usetikzlibrary{shapes.geometric, arrows, positioning, fit, calc, backgrounds}

% --- CODE STYLING ---
\lstset{
  basicstyle=\ttfamily\scriptsize,
  frame=single,
  breaklines=true,
  captionpos=b,
  numbers=left,
  numberstyle=\tiny\color{gray},
  keywordstyle=\color{blue},
  commentstyle=\color{green!50!black},
  stringstyle=\color{red},
  xleftmargin=2em,
  framexleftmargin=1.5em
}

\def\BibTeX{{\rm B\kern-.05em{\sc i\kern-.025em b}\kern-.08em
    T\kern-.1667em\lower.7ex\hbox{E}\kern-.125emX}}

\begin{document}

% ==========================================
% TITLE
% ==========================================
\title{Cross-Campus Federated Intelligence: Scaling Privacy-Preserving Edge Learning Across Distributed Academic Institutions}

% ==========================================
% AUTHOR
% ==========================================
\author{\ IEEEauthorblockN{Premkumar Tatapudi}
\IEEEauthorblockA{\textit{Department of Computer Science} \\
\textit{ScholarMaster Research Group}\\
Andhra Pradesh, India \\
premkumartatapudi@example.com}
}

\maketitle

% ==========================================
% NOTE: FUTURE WORK DRAFT
% ==========================================
% This is a DRAFT manuscript for Paper 14 (Future Work)
% To be developed AFTER Papers 11-13 are published
% Key concepts to include in final version:
% - Hierarchical FedAvg (H-FedAvg) algorithm
% - Staleness-aware asynchronous aggregation
% - Cross-domain generalization experiments (5 campuses)
% - Governance layer for institutional data sovereignty
% - Bandwidth efficiency analysis
% - Energy consumption modeling
%
% Development Timeline:
% - Start after Paper 13 acceptance (6-12 months)
% - Requires multi-institution collaboration agreement
% - Needs IRB approval for cross-campus data collection
% ==========================================

% ==========================================
% ABSTRACT (DRAFT)
% ==========================================
\begin{abstract}
[DRAFT - To be refined after Papers 11-13 publication]

As academic institutions increasingly adopt AI-driven analytics, a critical "Silo Problem" emerges: models trained on a specific campus fail to generalize to other institutions due to divergent architectural layouts, demographic distributions, and environmental conditions (Non-IID data). Centralized data aggregation to solve this is legally and ethically prohibitive under regulations like GDPR and DPDP. This paper extends prior work on single-campus deployment by introducing a **Hierarchical Federated Learning (H-FedAvg)** architecture designed for cross-institutional collaboration. We propose a three-tier topology (Classroom $\rightarrow$ Campus Aggregator $\rightarrow$ Global Federation) that enables institutions to collaboratively train robust engagement and safety models without sharing raw student data.

[TODO: Add final simulation results, citation to Papers 11-13 as foundation]
\end{abstract}

\begin{IEEEkeywords}
Federated Learning, Hierarchical Aggregation, Cross-Domain Generalization, Privacy-Preserving AI, Institutional Governance, Non-IID Data, Smart Campus.
\end{IEEEkeywords}

% ==========================================
% DEVELOPMENT NOTES
% ==========================================
% Key contributions to develop:
% 1. Hierarchical FedAvg algorithm formalization
% 2. Staleness-aware asynchronous protocol
% 3. Multi-domain simulation (5+ synthetic campuses)
% 4. Governance framework for data sovereignty
% 5. Comparative analysis: Local-only vs Centralized vs Federated
%
% Experimental requirements:
% - Simulate 5 campuses with distinct characteristics
% - Measure cross-domain generalization (18% improvement target)
% - Bandwidth overhead analysis (<10% overhead target)
% - Client dropout resilience testing (40% dropout scenario)
%
% Related work to cite:
% - Papers 11-13 (this series) as foundation
% - McMahan et al. (FedAvg original)
% - Li et al. (FedProx for system heterogeneity)
% - Bonawitz et al. (Secure aggregation)
% ==========================================

\section{Introduction}
[TODO: Develop after Papers 11-13 acceptance]

% Placeholder structure:
% - Problem: Domain shift across institutions
% - Gap: No prior work on cross-campus biometric FL
% - Solution: Hierarchical aggregation with governance
% - Contributions: H-FedAvg, staleness protocol, sovereignty layer

\section{Future Development Plan}

This draft manuscript represents planned future work building upon Papers 11-13. Development will proceed in the following phases:

\subsection{Phase 1: Foundation (Complete)}
\begin{itemize}
    \item Paper 11: Single-campus deployment infrastructure
    \item Paper 12: Flash storage optimization for edge training
    \item Paper 13: Intra-campus federated learning for drift
\end{itemize}

\subsection{Phase 2: Cross-Campus Extension (6-12 months)}
\begin{itemize}
    \item Formalize Hierarchical FedAvg algorithm
    \item Develop staleness-aware asynchronous protocol
    \item Create multi-domain simulation environment
    \item Design governance layer for data sovereignty
\end{itemize}

\subsection{Phase 3: Deployment \& Validation (12-18 months)}
\begin{itemize}
    \item Partner with 3-5 pilot universities
    \item Deploy hierarchical aggregation infrastructure
    \item Collect real cross-campus generalization data
    \item Validate bandwidth/energy efficiency claims
\end{itemize}

\subsection{Phase 4: Publication (18-24 months)}
\begin{itemize}
    \item Submit to top-tier venue (MLSys, OSDI, ATC)
    \item Target acceptance 2-3 years post-Paper 13
\end{itemize}

% ==========================================
% REFERENCES (PRELIMINARY)
% ==========================================
\begin{thebibliography}{00}

\bibitem{p11} Premkumar Tatapudi, "From Lab to Lecture Hall: Production-Grade Edge MLOps Architecture,'' \textit{Paper 11 (this series)}, 2026.

\bibitem{p12} Premkumar Tatapudi, ``Flash Endurance Engineering for Edge AI,'' \textit{Paper 12 (this series)}, 2026.

\bibitem{p13} Premkumar Tatapudi, ``Privacy-Preserving Federated Learning for ModelDrift,'' \textit{Paper 13 (this series)}, 2026.

\bibitem{b1} B. McMahan et al., ``Communication-Efficient Learning of Deep Networks from Decentralized Data,'' \textit{AISTATS}, 2017.

\bibitem{b2} T. Li et al., ``Federated Optimization in Heterogeneous Networks,'' \textit{MLSys}, 2020.

\bibitem{b3} K. Bonawitz et al., ``Practical Secure Aggregation for Privacy-Preserving Machine Learning,'' \textit{ACM CCS}, 2017.

% [TODO: Add 10-15 more references for final version]

\end{thebibliography}

\end{document}
